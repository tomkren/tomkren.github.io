\documentclass{article}

\usepackage{xspace}


\newcommand{\lterm}{$\lambda$-term\xspace}
\newcommand{\lterms}{$\lambda$-terms\xspace}
\newcommand{\ro}[1]{{\color{blue} #1}}


\begin{document}


\section{Related work}
\label{related}

\ro{
In \cite{yu01} Yu presents a GP system utilizing
polymorphic higher-order functions\footnote{Higher-order 
function is a function taking another function as 
input parameter.} and lambda abstractions.
Important point of interest in this work is use of
\texttt{foldr} function as a tool for \textit{implicit recursion},
i.e. recursion without explicit recursive calls. 
The terminal set for constructing lambda abstraction subtrees 
is limited to use only constants and variables of that particular
lambda abstraction, i.e., outer variables are not allowed to be used
as terminals in this work. This is significant difference from our approach 
since we permit all well-typed normalized \lterms. From this difference also
comes different crossover operation. We focus more on term generating process; 
their term generation is performed in a similar way as the standard one, 
whereas our term generation also tries to utilize techniques of systematic enumeration. 
}
In \cite{kes} Briggs and O’Neill present technique 
utilizing typed GP with combinators.
The difference between approach presented in this work
and our approach is that in this work terms are generated
straight from \textit{library} of combinators and no lambda abstractions
are used. They are using more general polymorphic type system then us
-- the Hindley–Milner type system. They also discuss the 
properties of exhaustive enumeration of terms and compare it with GP search.  
They also present interesting concept of \textit{Generalized
genetic operator} based on term generation. 

In \cite{binard2008genetic} by Binard and Felty even 
stronger type system (\textit{System F}) is used.  
But with increasing power of the type system comes increasing difficulty of term generation.
For this reason evolution in this work takes interesting and nonstandard shape 
(fitness is associated with \textit{genes} which are evolved together with \textit{species}
which together participate in creation of individuals).
This differs from our approach, which tries to be generalization of
the standard GP\cite{koza92}.

In contrast with above mentioned works our approach uses very simple type system 
(simply typed lambda calculus) and concentrates on process of generation  
able to generate all possible well-typed normalized lambda terms. In order to do
so we use technique based on \textit{inhabitation machines} 
described by Barendregt in~\cite{barendregt10}.    


\begin{thebibliography}{1}


\bibitem{koza92}
  John R. Koza,
  \emph{Genetic Programming: On the Programming of Computers by Means of Natural Selection}.
  MIT Press, Cambridge, MA,
  1992. 

\bibitem{yu01}
  T. Yu. 
  \emph{Hierachical processing for evolving recursive and modular 
        programs using higher order functions and lambda abstractions}. 
  Genetic Programming and Evolvable Machines,
  2(4):345–380, December 2001. ISSN 1389-2576.

\bibitem{binard2008genetic}
Binard, F., Felty, A.: Genetic programming with polymorphic types and higher-order 
functions. In: Proceedings of the 10th annual conference on Genetic and
evolutionary computation, ACM (2008) 1187-1194

\bibitem{kes}
Forrest Briggs, Melissa O’Neill.
\emph{Functional Genetic Programming and Exhaustive
Program Search with Combinator Expressions.}
International Journal of Knowledge-based and Intelligent Engineering Systems,
Volume 12 Issue 1, Pages 47-68, January 2008. 


\bibitem{barendregt10}
  Henk Barendregt, Wil Dekkers, Richard Statman,
  \emph{Lambda Calculus With Types}.
  Cambridge University Press,
  2010. 
  %URL \url{http://www.cs.ru.nl/~henk/book.pdf}.


\end{thebibliography}

\end{document}