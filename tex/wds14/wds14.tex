%to be compiled with LATEX 2e

%use this setting with 11pt basic font:
\documentclass[11pt]{article}
\usepackage{wds11,epsf}

%use this setting with 10pt basic font:
%\documentclass{article}
%\usepackage{wds10,epsf}

%use to get author-year citations with BibTeX
\usepackage[square]{natbib}





\usepackage{xspace}


\newcommand{\Lets}{Let us\xspace}
\newcommand{\lets}{let us\xspace}
\newcommand{\lterm}{$\lambda$-term\xspace}
\newcommand{\lterms}{$\lambda$-terms\xspace}
\newcommand{\lhead}{$\lambda$-head\xspace}
\newcommand{\lheads}{$\lambda$-heads\xspace}
\newcommand{\myla}{\leftarrow\xspace}
\newcommand{\Lp}  {\Lambda^{\prime}\xspace}
\newcommand{\tur}[3]{#1\vdash{}#2 \colon #3}
\newcommand{\turst}[3]{$#1\vdash{}#2:#3$\xspace}
\newcommand{\GMS}{\turst{\Gamma}{M}{\sigma}}
\newcommand{\atTree}{@-tree\xspace}
\newcommand{\setDots}[2]{ \lbrace #1 , \dots , #2 \rbrace}
\newcommand{\lh}[1]{\lambda #1}
\newcommand{\sexprTree}{sexpr-tree\xspace}
\newcommand{\SexprTree}{Sexpr-tree\xspace}
\newcommand{\then}{\Rightarrow\xspace}
\newcommand{\lamb}[2]{( \lambda \, #1 \, . \, #2 )}
\newcommand{\lam}[2]{\lambda \, #1 \, . \, #2}
\newcommand{\ST}{\mathop{\mathrm{ST}}}
\newcommand{\FV}{\mathop{\mathrm{FV}}}
\newcommand{\Scomb }{\mathbf{S}}
\newcommand{\Kcomb }{\mathbf{K}}
\newcommand{\Icomb }{\mathbf{I}}
\newcommand{\bbarr}{\twoheadrightarrow_\beta}
\newcommand{\barr}{\rightarrow_\beta}
\newcommand{\beq}{=_\beta}
\newcommand{\eearr}{\twoheadrightarrow_\eta}
\newcommand{\earr}{\rightarrow_\eta}
\newcommand{\eeq}{=_\eta}
\newcommand{\bearr}{\rightarrow_{\beta\eta}}
\newcommand{\bbeearr}{\twoheadrightarrow_{\beta\eta}}
\newcommand{\beeq}{=_{\beta\eta}}
\newcommand{\etar}{\twoheadrightarrow_\eta}
\newcommand{\ered}{$\eta$-reduction\xspace}
\newcommand{\bnf}{$\beta$-\textit{nf}\xspace}
\newcommand{\enf}{$\eta$-\textit{nf}\xspace}
\newcommand{\eenf}{$\eta^{-1}$-\textit{nf}\xspace}
\newcommand{\beenf}{$\beta\eta^{-1}$-\textit{nf}\xspace}
\newcommand{\benf}{$\beta\eta$-\textit{nf}\xspace}
\newcommand{\bredex}{$\beta$-redex\xspace} 
\newcommand{\lnf}{\textit{lnf}\xspace}
\newcommand{\Ae}{\mathop{\mathrm{\AE}}}
\newcommand{\Bcomb }{\mathbf{B}}   
\newcommand{\BBcomb }{\mathbf{B*}}
\newcommand{\Ccomb }{\mathbf{C}}   
\newcommand{\CCcomb }{\mathbf{C'}}
\newcommand{\SScomb }{\mathbf{S'}}
\newcommand{\ar}{\rightarrow\xspace}
\newcommand{\T}{\mathbb{T}\xspace}
\newcommand{\C}{\mathbb{C}\xspace}
\newcommand{\Real}{\mathbb{R}}

\newenvironment{todo}
{~\\ {\color{red}\textbf{TODO}}
  \begin{easylist}[itemize]}
{ \end{easylist}}

\newcommand{\Lpr}{\Lambda^\prime}
\newcommand{\ul}[2]{\langle #1 ; #2 \rangle}
\newcommand{\ro}[1]{{\color{blue} #1}}
\newcommand{\tom}[1]{{\color{ForestGreen} #1}}
\newcommand{\red}[1]{{\color{red} #1}}




%\tighten

% Preamble Information


%%%%%%%%%%% nevim co to tu bylo za omg, tak jsem to zakomentoval
%%\lefthead{SAFRANKOVA ET AL.}
%%\righthead{MAGNETOSHEATH RESPONSE}

\setcounter{secnumdepth}{0}

\begin{document}

\title{Typed Genetic Programming over Lambda Calculus}

\author{T. K\v ren}

\affil{Charles University, Faculty  of  Mathematics  and  Physics,
     Prague, Czech Republic.}


\begin{abstract}
Todo.
\end{abstract}

\begin{article}

\section{Introduction}

\section{Related work}
\label{related}

\cite{yu01} presents a GP system utilizing
polymorphic higher-order functions\footnote{Higher-order 
function takes another function as an input parameter.
} and lambda abstractions.
Important point of interest in this work is use of \texttt{foldr}\footnote{ In the functional programming language Haskell \texttt{foldr} can be defined as:\\ \texttt{foldr f z [] $~~~$  = z\\ 
foldr f z (x:xs) = f x (foldr f z xs) }} function as a tool for \textit{implicit recursion},
i.e. recursion without explicit recursive calls. 
The terminal set for constructing lambda abstraction subtrees 
is limited to use only constants and variables of that particular
lambda abstraction, i.e., outer variables are not allowed to be used
as terminals in this work. This is significant difference from our approach 
since we permit all well-typed normalized \lterms. From this difference also
comes different crossover operation. We focus more on term generating process; 
their term generation is performed in a similar way as the standard one, 
whereas our term generation also tries to utilize techniques of systematic enumeration. 

\cite{kes} present technique 
utilizing typed GP with combinators.
The difference between approach presented in this work
and our approach is that in this work terms are generated
straight from \textit{library} of combinators and no lambda abstractions
are used. They are using more general polymorphic type system than us
-- the Hindley–Milner type system. They also discuss the 
properties of exhaustive enumeration of terms and compare it with GP search.  
They also present interesting concept of \textit{Generalized
genetic operator} based on term generation. 

\cite{binard2008genetic} use even 
stronger type system (\textit{System F}).  
But with increasing power of the type system comes increasing difficulty of term generation.
For this reason evolution in this work takes interesting and nonstandard shape 
(fitness is associated with \textit{genes} which are evolved together with \textit{species}
which together participate in creation of individuals).
This differs from our approach, which tries to be generalization of
the standard GP\cite{koza92}.

In contrast with above mentioned works our approach uses very simple type system 
(simply typed lambda calculus) and concentrates on process of generation  
able to generate all possible well-typed normalized lambda terms. In order to do
so we use technique based on \textit{inhabitation machines} 
described by Barendregt \cite{barendregt10}.    




%%% Abych věděl jak citovat tak sem si tu kousek nechal
%The interaction of the solar wind with the Earth's magnetosphere
%generates a population of backstreaming ions directed from the
%bow shock into the solar wind. This high-energy population was
%invoked to explain Hot Flow Anomalies (HFAs)
%\citep[e.g.][]{sw}. \citet{th} identified as heated regions of solar wind plasma
%flowing nearly perpendicular to the Earth-Sun line.
%The main observational features of HFAs include \citep[and others]{sw}: (1) Central regions with hot plasma flowing
%significantly slower than that in the ambient solar wind in a
%direction highly deflected (nearly $90^o$) from the Sun-Earth line.
%The flow velocities are often roughly tangential to the nominal
%bow shock shape \citep{sw}.
%(2) HFAs are bounded by regions of enhanced
%magnetic field strength, density, and temperature. The outer
%edges of these enhancements are fast shocks generated by
%pressure enhancements within the core region. The inner
%edges of the enhancements are probably  tangential
%discontinuities.
%Published examples indicated that  many HFAs are bounded by only
%one enhancement.
%(3) HFAs occur in conjunction with significant changes in the IMF
%direction. The angle between pre- and post event orientations is
%typically $\sim 70^o$.

Todo.

\section{Conclusion}
Todo.


\acknowledgments %\footnotesize
{???}


%% TWO METHODS FOR INCLUDING THE BIBLIOGRAPHY (LIST OF REFERENCES)
%% EITHER TYPE IN THE ENTRIES YOURSELF AS SHOWN HERE IN
%% `thebibliography' ENVIRONMENT,
%%        OR
%% USE THE FOLLOWING TWO COMMANDS SO THAT BIBTEX WILL GENERATE
%% `thebibliography' TEXT FOR YOU AND READ IT IN.
%%
\bibliographystyle{egs}%<-- LIST OF REFERENCES TO BE IN "EGS" STYLE (FILE "EGS.STY")
\bibliography{my}       %<-- REFERENCES ARE IN FILE "SAMPLE.BIB"
%%
%%  IF THE ABOVE TWO COMMANDS ARE USED, THEN thebibliography ENVIRONMENT
%%  MUST BE REMOVED.

%\begin{thebibliography}{}
%
%\itemsep=0.4ex
%%\footnotesize
%\bibitem[\it Klimov and Friend(1997)]{kl}
%Klimov, S., Friend A., ASPI experiment: Measurements of fields and
%waves onboard the Interball-1 spacecraft, {\it Ann. Geophys.}, 15,
%514-527, 1997.
%
%\bibitem[\it Kudela et al.(1995)]{ku}
%Kudela, K., M. Slivka, J. Rojko, and V. N. Lutsenko, The
%apparatus DOK-2 (project INTERBALL): Output data structure and
%modes of operation, preprint of Inst. Exp. Phys. UEF-01-95,
%Kosice, 20, 1995.
%
%\bibitem[\it Lin(1997)]{li}
%Lin Y., Generation of anomalous flows near the bow shock by
%its interaction with interplanetary discontinuities, {\it J. Geophys.
%Res.}, 102, 24265-24281, 1997.
%
%\bibitem[\it Lutsenko et al.(1995)]{lu}
%Lutsenko, V. N., J. Rojko, K. Kudela, T. V. Gretchko, J. Balaz,
%J. Matisin, E. T. Sarris, K. Kalaitzides, and N. Paschalidis,
%Energetic particle experiment DOK-2 (Interball project), in:
%{it INTERBALL Mission and Payload,} ed. by Yu. Galperin, IKI-CNES,
%249, 1995.
%
%
%\bibitem[\it Onsager et al.(1990a)]{ona}
%Onsager, T. G., M. F. Thomsen, J. G. Gosling, and S. J.
%Bame, Observational test of a hot flow anomaly formation
%mechanism, {\it J. Geophys. Res.}, 95, 11.967-11.974, 1990a.
%
%\bibitem[\it Onsager et al.(1990b)]{onb}
%Onsager, T. G., M. F. Thomsen, and D. Winske,
%Hot flow anomaly formation by magnetic deflection,
%{\it Geophys. Res. Lett.}, 17, 1621-1624, 1990b.
%
%\bibitem[\it Schwartz et al.(1999d)]{sw}
%Schwartz, S. J., G. E. Paschmann, N. Sckopke, T. M. Bauer, M. W.
%Dunlop, A. N. Fazakerley, and M. F. Thomsen,  Hot flow anomalies
%revisited, {\it Int. J. Geomag. and Aeronomy}, 1, 1999, in print.
%
%\bibitem[\it Thomsen et al.(1992)]{th}
%Thomsen M. F., J. G. Gosling, and S. J.
%Bame, Observational test of a hot flow anomaly formation
%mechanism, {\it J. Geophys. Res.}, 97, 1967-1974, 1992.
%
%
%\end{thebibliography}


\end{article}
\end{document}

